Magnetic Resonance Imaging (MRI) of the brain has been used to investigate a wide range of neurological disorders, but data acquisition can be expensive, time-consuming, and inconvenient. Multi-site studies present a valuable opportunity to advance research by pooling data in order to increase sensitivity and statistical power. However images derived from MRI are susceptible to both obvious and non-obvious differences between sites which can introduce bias and subject variance, and so reduce statistical power. To rectify these differences, we propose a data driven approach using a deep learning architecture known as generative adversarial networks (GANs). GANs learn to estimate two distributions, and can then be used to transform examples from one distribution into the other distribution. Here we transform T1-weighted brain images collected from two different sites into MR images from the same site. We evaluate whether our model can reduce site-specific differences without loss of information related to gender (male or female) or clinical diagnosis (schizophrenia or healthy). When trained appropriately, our model is able to normalise imaging sets to a common scanner set with less information loss compared to current approaches. An important advantage is our method can be treated as a `black box'  that does not require any knowledge of the sources of bias but only needs at least two distinct imaging sets.
